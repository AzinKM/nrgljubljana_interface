\documentclass[prb,amsmath,amssymb,onecolumn,nopacs,floatfix]{revtex4}
\usepackage{graphicx}
\usepackage[english]{babel}
\usepackage{times}
\usepackage{dcolumn}
\usepackage[usenames,dvipsnames]{color}
\usepackage{units}
\usepackage{color}

\begin{document}

\title{Hilbert transform (NRG Ljubljana and TRIQS implementation)}

\author{Rok \v{Z}itko}

\date{\today}

\begin{abstract}
\end{abstract}

\maketitle

\newcommand{\vc}[1]{{\mathbf{#1}}}
\newcommand{\vck}{\vc{k}}
\newcommand{\braket}[2]{\langle#1|#2\rangle}
\newcommand{\expv}[1]{\langle #1 \rangle}
\newcommand{\corr}[1]{\langle\langle #1 \rangle\rangle}
\newcommand{\ket}[1]{| #1 \rangle}
\newcommand{\bra}[1]{\langle #1 |}
\newcommand{\Tr}{\mathrm{Tr}}
\newcommand{\kor}[1]{\langle\langle #1 \rangle\rangle}
\newcommand{\degg}{^\circ}
\renewcommand{\Im}{\mathrm{Im}}
\renewcommand{\Re}{\mathrm{Re}}
\newcommand{\dtN}{{\dot N}}
\newcommand{\dtQ}{{\dot Q}}
\newcommand{\sgn}{\mathrm{sgn}}
\newcommand{\beq}[1]{\begin{equation} #1 \end{equation}}
\newcommand{\one}{\mathbf{1}}

\section{Definition}

The Hilbert transform of (spectral) function $A(\omega)$ is defined as
%
\beq{
H(z) = \int_{-\infty}^{\infty} \frac{A(\omega)}{z-\omega} d\omega.
}
%
The argument is a complex number $z=x+i y$. For large $|y|$, this integral
can be evaluated directly. The tabulated spectral function $A(\omega)$ is
interpolated using cubic splines (GSL). Outside its domain of definition (tabulation
range), the spectral function is taken to be equal to zero.
The real and imaginary parts of the result 
are computed separately using real-valued integration:
%
\beq{
\begin{split}
\Re H(z) &= \int_{-\infty}^{\infty} \frac{\Re A(\omega) (x-\omega) + \Im A(\omega) y}
{(x-\omega)^2 + y^2} d\omega, \\
\Im H(z) &= \int_{-\infty}^{\infty} \frac{\Re A(\omega) (-y) + \Im A(\omega) (x-\omega)}
{(x-\omega)^2 + y^2} d\omega.
\end{split}
}
%
We use 15-point Gauss-Kronrod rule ({\tt GSL\_INTEG\_GAUSS15} argument to {\tt gsl\_integration\_qag}).
The accuracy goal is $10^{-14}$ absolute accuracy and $10^{-10}$ relative
accuracy; failure to reach the goal is not considered an error and does not
terminate the program.

For small $|y|$, we handle the singular point explicitly, by subtraction:
%
\beq{
H(z) = \int_{-\infty}^{\infty} \frac{A(\omega)-A(x)}{z-\omega} d\omega
+ A(x) \int_{-\infty}^{\infty} \frac{1}{z-\omega}d\omega.
}
%
The second integral, $Q(z)=\int d\omega/(z-\omega)$, is evaluated analytically and expressed
in terms of $\arctan$ and $\log$ functions. The first one is evaluated numerically after 
a change of variables, $\omega = x + |y| W$, $W=\exp(r)$ or $W=-\exp(r)$, with $r$ 
ranging from $\log(10^{-16})$ to the boundary of the integration range, 
on both sides of the singularity. The ensures very high accuracy of the result.

\section{Interfaces}

For scalar-valued $A(\omega)$, there are three interfaces:
%
\begin{itemize}
\item {\tt hilbert\_transform(A, z)} for single-point calculations,

\item {\tt hilbert\_transform(A, mesh, eps = 1e-16)} for calculations on
a mesh of real values with $y=\mathrm{eps}$, which produces the
retarded Green's function on the real-frequency axis, (the corresponding advanced
GF can be obtained using {\tt eps = -1e-16}),

\item {\tt hilbert\_transform(A, G)} for calculations on a set of 
complex values contained in a Green's function object $G(\omega)$, i.e., the
output is $H[G(\omega)]$. This is useful in the context of DMFT where
one needs to evaluate $H[\omega+\mu-\Sigma(\omega)]$.
\end{itemize}

For matrix-valued $A(\omega)$, there is an element-wise
{\tt hilbert\_transform\_elementwise} which threads over the matrix components.

For block GFs, there are two interfaces, one for single-point calculations and one for
calculations one a mesh.



\end{document}
